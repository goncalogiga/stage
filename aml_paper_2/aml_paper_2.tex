\documentclass[a4paper, 11pt]{article}
\usepackage[utf8]{inputenc}
\usepackage{amsmath}
\usepackage{array}
\usepackage{amssymb} 
\usepackage{array,multirow,makecell}
\usepackage{comment}
\usepackage{fullpage} 
\usepackage{graphicx}
\usepackage{bussproofs}
\usepackage{mathtools}
\usepackage[makeroom]{cancel}
\graphicspath{ {./images/} }
\usepackage{graphicx}
\usepackage{subfig}
\usepackage{tikz} 
\usetikzlibrary{positioning}

\usepackage{tcolorbox}
\usepackage{listings}
\usepackage{xcolor}
 
\definecolor{codegreen}{rgb}{0,0.6,0}
\definecolor{codegray}{rgb}{0.5,0.5,0.5}
\definecolor{codepurple}{rgb}{0.58,0,0.82}
\definecolor{backcolour}{rgb}{0.95,0.95,0.92}
 
\lstdefinestyle{mystyle}{
    backgroundcolor=\color{backcolour},   
    commentstyle=\color{codegreen},
    keywordstyle=\color{magenta},
    numberstyle=\tiny\color{codegray},
    stringstyle=\color{codepurple},
    basicstyle=\ttfamily\footnotesize,
    breakatwhitespace=false,         
    breaklines=true,                 
    captionpos=b,                    
    keepspaces=true,                 
    numbers=left,                    
    numbersep=5pt,                  
    showspaces=false,                
    showstringspaces=false,
    showtabs=false,                  
    tabsize=2
}
 
\lstset{style=mystyle}

\newtcolorbox{mybox}[3][]
{
  colframe = #2!25,
  colback  = #2!10,
  coltitle = #2!20!black,  
  title    = {#3},
  #1,
}

\title{Finite Atomized Semilattices - Notes}

\begin{document}
\maketitle

\section{Preliminaries}

\subsection{Semilllatice}

\begin{tcolorbox} 
	\textbf{Definition:} A \textit{semillatice} is an algebra $(M,\odot)$ where $\odot$ is a commutative, associative and idempotent binary operation.
\end{tcolorbox}

From this very simple definition we can already define a partial order we denote $\leq$. Indeed, a partial order is a binary relation $R$ that is reflexive, antisymetric and transitive. All these propreties derive from the commutativity, associativity and idempotence of $\odot$:

\

Let us define $\leq$ such that $a\leq b$ iff $a \odot b = b$.

\

The idempotence proprety gives us the reflexivity $a \leq a = a \odot a = a$.  

\

From the commutativity of $\odot$ we get: $a \odot b = b \odot a$. So if $a \leq b$ and $b \leq a$ and therefore $a \odot b = b$ and $b \odot a = a$, then $a=b$.

\

From the associativity of $\odot$: $a \odot (b \odot c) = (a \odot b) \odot c$; so if $a \leq b$ and $b \leq c$ we have $a \odot b = b$ and $b \odot c = c$; so $a \odot (b \odot c) = a \odot c = (a \odot b) \odot c = b \odot c = c$. Therefore we've got $a \leq c$ given that we prooved $a \odot c = c$. 

\

\begin{tcolorbox} 
	\textbf{Definition:} We say $a \leq b$ in a semilattice $M$ if and only if $M \Vdash (b = b \odot a)$.
\end{tcolorbox}

\subsection{Terms, constants and atoms}

\subsubsection{In the first paper}

The first paper describing AML uses semilattices to embed problems into an algebra. The binary operation is called merge (symbol $\odot$) and the set $S$ has three types of of elements: constants, terms and atoms.

\

Terms should describe the objects of the problem. For exemple, terms can represent images as matrices. Here's a simple exemple of a 4-pixel image described by a term $T$:

\begin{equation*}
    T = \begin{bmatrix}
    c_1 & c_2 \\
    c_3 & c_4
    \end{bmatrix}
\end{equation*}

Here $c_1, c_2, c_3$ and $c_4$ are constants, the primitive description elements of any embedding problem, and so the previous representation of $T$ in the algebra $S$ is:

\begin{equation*}
    T = c_1 \odot c_2 \odot c_3 \odot c_4
\end{equation*}

Atoms are elements created by the learning algorithm, represented by greek letters, and each constant is a merge of atoms. Here, $T$ is therefore a merge of atoms too.

\subsubsection{In the second paper}

In the second paper, the elements of the free algebra $F_C(\emptyset)$ are called terms. A duple $r = (r_L, r_R)$ is defined as an ordered pair of elements of $F_C(\emptyset)$. Positive and negative duples, written $r^+$ and $r^-$, are used as follows: we say $M \Vdash r^+$ iff $M \Vdash r_L \leq r_R$ and we say $M \Vdash r^-$ iff $M \not\Vdash r_L \leq r_R$. 

\

A semilattice model $M$ is said to be \textit{atomized} when it is extended with a set of additional elements called atoms. Each semilattice is a partial order and an atomization of the semillatice is an extension of the partial order. However, an atomized semilattice is not a semilattice extension; meaning in atomized semilattices the idempotent operator is defined exclusively for regular elements while the order relation is defined for all, atoms and regular elements. 
Formaly, this means that $\leq$ is defined for all elements of the atomized semilattice but $\odot$ is restricted to the regular elements of the atomized semilattice..

\

Given that $\leq$ is well defined for all elements of the atomized semilattice; it is therefore a \textbf{partially ordered set}. Let us formalize this into a definition.

\subsection{Formal definition of an atomized semilattice}

\

\begin{tcolorbox} 
	\textbf{Definition:} An atomized semilattice over a set of constants $C$ is a structure $M$ with elements of two sorts, the regular elements $\{a,b,c,...\}$ and the atoms $\{\phi, \psi, ...\}$, with an itempotent, commutative and associative binary operator $\odot$ defined for regular elements and a partial order relation $<$ that is defined for all elements, such that the regular elements are either constants or idempotent summations of constants, and M satisfies the axioms of the operations and the additional:

\

	AS1: $$\forall \phi \exists c, (c \in C) \land (\phi < c)$$
	AS2: $$\forall \phi \forall a (a \not \leq \phi)$$
	AS3: $$\forall a \forall b (a \leq b \Leftrightarrow \neg \exists \phi, ((\phi < a) \land (\phi \not < b))$$
	AS4: $$\forall \phi \forall a \forall b (\phi < a \odot b \Leftrightarrow (\phi < a) \lor (\phi < b))$$
	AS5: $$\forall c \in C ((\phi < c) \Leftrightarrow (\psi < c)) \Rightarrow (\phi = \psi)$$
	AS6: $$\forall a \exists \phi, (\phi < a)$$
\end{tcolorbox}

\colorbox{yellow}{Notes}

\begin{itemize} 
	\item AS1 states that for each atom, there exists a constant in $C$ edeged to it. So every atom is edeged to at least one constant. ie; there are no "hanging" atoms.	
	\item AS2 states that atoms are at the bottom of the relation order (if we visualize the order relation in a graph)
	\item AS3 states that for every couple of regular elements, if $b$ is higher than $a$, then there is no atom that is edged to $a$ but not to $b$; ie. if $a \leq b$, all atoms ,edeged to $a$ are also edged to $b$.
	\item AS4 states the following: every atom edeged to the merge of $a$ and $b$ must be edged to either $a$ or $b$.
	\item AS5 states that every couple of atoms that are edged to the same constants must be equal.
	\item AS6 states that each regular element is edged to at least one atom.
\end{itemize}

\


Here, $<$ represents the partial order relation between any element of the atomized semilattice. In this formal definiton, $\leq$ is the partial order relation defined from  the operator $\odot$ like it was shown in section 1.1

\

However, $a \leq b \Leftrightarrow a \odot b = b$ is not an axiom and it should be verified that $\leq$ still holds. Let's define a new formula called AS3b: $\forall a \forall b (a \leq b \Leftrightarrow a \odot b = b)$.

\

\begin{mybox}{red}{\textbf{Theorem 1}}
	Assume AS4 and the antisymmetry of the order relation.
	\begin{itemize} 
		\item i) AS3 implies AS3b
		\item ii) Assume $\forall a \forall b (\forall \phi (( \phi < a ) \Leftrightarrow (\phi < b)) \Rightarrow (a = b))$. Then AS3b $\Rightarrow$ AS3.
		\item iii) AS3 implies $\forall a \forall b ( \forall \phi ((\phi < a) \Leftrightarrow (\phi < b)) \Rightarrow (a = b))$.
	\end{itemize}
\end{mybox}

\noindent
\underline{Proof for i):} Let $a$ and $b$ be regular elements of an atomized semilattice. Let's suppose that $a \leq b$ holds. From AS3 we get $\neg \exists \phi, ((\phi < a) \land (\phi \not < b))$. Developping the negation and using the logical implication gives us the expression $\forall \phi ((\phi < a) \Rightarrow (\phi < b))$ which states very clearly that each atom edged to $a$ must also be edged to $b$.

\

Therefore we have $(\phi < a \lor \phi < b \Leftrightarrow \phi < b)$. Indeed, if we imagine a tree proof, the only proposition that wouldn't be imediatly true is $\phi < a \Rightarrow \phi < b$ but we just showed this is also true.

\

Given the last proposition, AS4 becomes $(\phi < a \odot b \Leftrightarrow \phi < b)$. The antisymmetry of $<$ gives us a way of getting $a \odot b = b$ if we proove $(a \odot b < b) \land (b < a \odot b)$. This can be done by splitting the equivalence we got from AS4 into $(\phi < a \odot b) \Rightarrow (\phi < b)$ and $(\phi < b) \Rightarrow (\phi < a \odot b)$. By using AS3 from right to left we get both $a \odot b < b$ and $b < a \odot b$, giving us the desired result.

\

Now let's assume that $a \odot b = b$. From AS4 we get $(\phi < b \Leftrightarrow (\phi < a) \lor (\phi < b))$ so $\phi < a \Rightarrow \phi < b$. Applying AS3 directly gives us $a \leq b$.

\

\noindent
\underline{Proof of ii):} Let $a$ and $b$ be rergular elements of an atomized semilattice such that for each atom $\phi$: $((\phi < a) \Leftrightarrow (\phi <  b)) \Rightarrow a = b$. Let's assume that AS3b is true: $a \leq b \Leftrightarrow a \odot b = b$.

\

Assuming $a \leq b$, we want to proove $(\phi < a) \Rightarrow (\phi < b)$. From $a \leq b$, AS4 becomes: $(\phi < b) \Leftrightarrow (\phi < a) \lor (\phi < b)$; which gives us our proof.

\

Now we assume $(\phi < a) \Rightarrow (\phi < b)$ and we would like to prove $a \leq b$, wich is equivalent to $a \odot b = b$ given AS3b. From this first assumption we get $(\phi < a \lor \phi < b) \Leftrightarrow \phi < b$ (cf. previous proof) and so AS4 gives us the formula $\phi < a \odot b \Leftrightarrow \phi < b$. Using the formula we assumed $((\phi < p) \Leftrightarrow (\phi < q)) \Rightarrow p = q$ with $p = a \odot b$ and $q = b$ we get $a \odot b = b$.

\

\noindent
\underline{Proof of iii)}: Let's suppose AS3 for $a$ and $b$ in the set of regular elements: \\
$(a \leq b \Leftrightarrow \forall \phi (\phi < a) \Rightarrow (\phi < b))$ (we're using the formulation discussed in proof i.).

\

Assuming $(\phi < a) \Leftrightarrow (\phi < b)$ we have, from AS3 applied two times where only the order of $a$ and $b$ changes, $a \leq b$ and $b \leq a$. This gives us $a=b$ because of the antisymmetry proprety.

\

\

From theorem 1, AS3 and AS3b are equivalent given that we have AS4 and the additional axiom:

$$\forall a \forall b (\forall \phi (( \phi < a ) \Leftrightarrow (\phi < b)) \Rightarrow (a = b))$$

Since this axiom follow from AS3; the partial order of the atomized semilattices does coincide with the semillatice spawned by its regular elements given the axioms of the first definition.

\

\begin{tcolorbox} 
	\textbf{Proprety:} There is always a natural homomorphism $v_M$ from the term algebra $F_C(\emptyset)$ onto any model $M$ that maps terms to elements of $M$.
\end{tcolorbox}

\begin{tcolorbox} 
	\textbf{Definition:} The lower and upper segment of an element $x$ are defined as $L_M(x) =\{y: y < x \lor y = x\}$ and $U_M(x) =\{y : y > x\}$. The superscript $^a$ is used to denote the intersection with the atoms: $L_M^a(x) = L_M(x) \cap A(M)$. The superscript $^c$ is used to represent the intersection with the constants: $L_M^c(x) = L_M(x) \cap C(M)$ (symmetricly, we have $U_M^a$ and $U_M^c$).

	\

	In an atomized semillatice model, $U_M^c(\phi)$ is defined as the set of constants in the upper segment on the atom $\phi$. Since an atom is defined by the constants in its upper segment we often drop the subindex $M$ and simply write $U^c(\phi)$.
\end{tcolorbox} 
\

\begin{mybox}{red}{\textbf{Theorem 2}}
	Let $t,s \in F_C(\emptyset)$ be two terms that represent two regular elements $v_M(t)$ and $v_M(s)$ of an atomized model $M$ over a finite set of constants $C$. Let $\phi$ be an atom, $c$ a constant in $C$ and let $a$ be a regular element of $M$.

	\begin{itemize} 
		\item i) $\forall t \forall c ( c \in C(t) \Rightarrow v_M(c) \leq v_M(t))$.
		\item ii) $\phi < v_M(t) \Leftrightarrow \exists c: ((c \in C(t)) \land (\phi < v_M(c)))$.
		\item iii) $(\phi < a) \Leftrightarrow \exists c: ((c \in C) \land (\phi < v_M(c) \leq a))$
		\item iv) $L_M^a(v_M(t)) =\{\phi \in M : C(t) \cap U^c(\phi) \neq \emptyset\}$
		\item v) $L_M^a(v_M(s) \odot v_M(t)) = L_M^a(v_M(t)) \cup L_M^a(v_M(s))$
		\item vi) $v_M(t) \leq v_M(s) \Leftrightarrow L_M^a(v_M(t)) \subseteq L_M^a(v_M(s))$
	\end{itemize}
\end{mybox}

\noindent 
\underline{Proof of i):} Let $t$ be a term and $c$ a constant. Let's assume $c \in C(t)$; then $t = t \odot c$ given that $c$ is a constant of $t$. Using the definition of the homomorphism $v_M$ we immediatly get $v_M(t \odot c) = v_M(t) \odot v_M(c) = v_M(t)$ which means $v_M(c) \leq v_M(t)$ using AS3b.

\

\noindent
\underline{Proof of ii):} Let's assume $\phi < v_M(t)$. Given that C(t) contains at least one element, we know that $t = \bigodot_{i \geq 1} c_i$, for $c_i \in C(t)$. In addition, AS4 can be generalized like this for a finite $i$:

$$\phi < \bigodot_i a_i \Leftrightarrow \bigcup_i (\phi < a_i)$$

Using AS4 and the propreties of the homomorphism, we have $\phi < v_M(t) \Leftrightarrow \bigcup_i \phi < v_M(c_i)$ and given our first assumtion, the left term is true so the right one must also be true; ie. there must exist at least one constant (let's name it $c$) such that $\phi < v_M(c)$, making the big union true.

\

\noindent
\underline{Proof of iii):} By generalizing ii) to any term we get $(\phi < a) \Leftrightarrow \exists c : ((c \in C) \land (\phi < v_M(c)))$. Given that any term as at least one constant that can be merged into it, let's call it $c \in C$, it is also true that $(\phi < a) \Leftrightarrow \exists c : (c \in C) \land (\phi < v_M(c)) \land (v_M(c) \odot a = a)$. Applying AS3b finishes the proof.



\

\noindent
\underline{Proof of iv):} (Reminders) We know that $L_M^a(x) =\{y : y < x \lor y = x\} \cap A(M)$; so it is the set of atoms edged to $x$. We also know that $U^c(\phi) =\{y : y > x\} \cap C(M)$; so it is the set of constants that are edged to $\phi$. The intersection $C(t) \cap U^c(\phi)$ is therefore the constants edeged both to $\phi$ and $t$ (the middle layer so to speak). Each atom $\phi$ that verifies $C(t) \cap U^c(\phi) \neq \emptyset$ is therefore edged to each constants that are also edged to $t$.

\

Let $\phi \in L_M^a(v_M(t))$; then $\phi < v_M(t)$ and by ii. : $\exists c : ((c \in C(t) \land (\phi < v_M(c))$. This shows that $c$ is both in $C(t)$ and $U^c(\phi)$ because it is edged to $\phi$. Now let $c \in C(t) \cap U^c(\phi)$, from iii. we get $\phi < v_M(c) \leq v_M(t)$, meaning $\phi \in L_M^a(v_M(t))$. 

\

\noindent
\underline{Proof of v:} From iv. we have: $L_M^a(v_M(s \odot t)) =\{\phi \in M : C(s \odot t) \cap U^c(\phi) \neq \emptyset\}$. The constants of the merge of $s$ and $t$ are $C(s) \cup C(t)$ so we get $L_M^a(v_M(s \odot t)) =\{\phi \in M : (C(s) \cup C(t)) \cap U^c(\phi) \neq \emptyset\}$. By distribution we get: $L_M^a(v_M(s \odot t)) =\{\phi \in M : C(s) \cap U^c(\phi) \neq \emptyset\} \cup \{\phi \in M : C(t) \cap U^c(\phi) \neq \emptyset\}$. Finaly, using the proprety of the homomorphism and iv. we get: $L_M^a(v_M(s) \odot v_M(t)) = L_M^a(v_M(s)) \cup L_M^a(v_M(t))$.

\

\noindent
\underline{Proof of vi.} From AS3b, we know that $v_M(t) \leq v_M(s) \Leftrightarrow v_M(t) \odot v_M(s) = v_M(s)$. Using proprety v. we get $v_M(t) \leq v_M(s) \Leftrightarrow L_M^a(v_M(t)) \cup L_M^a(v_M(s)) = L_M^a(v_M(s))$ which gives us \\
$v_M(t) \leq v_M(t) \Leftrightarrow L_M^a(v_M(t)) \subseteq L_M^a(v_M(s))$ given that for any sets $A$ and $B$: $A \cup B = B \Leftrightarrow A \subseteq B$.

\

\

\

Axiom AS1 says that the upper constant segment of an atom is never empty while axiom AS5 says that two distinct atoms cannot have the same upper constant segment. In other words, an atom is always edged to a constant and to distinct atoms cannot be edged to the same constants.

\

Axiom AS3 implies that the order relation for regular elements is fully determined by the atoms of the model. 

\

Theorem 2. proposition (v) proves the linearity proprety:
$$ L_M^a(s \odot t) = L_M^a(s) \cup L_M^a(t)$$
where we have omitted the natural homomorphism.

\

Theorem 2. proposition (vi) and AS3b imply that:
$$M \Vdash (t = s) \Leftrightarrow L_M^a(t) = L_M^a(s)$$

Meaning it does not matter which term we use to represent a regular element; we get the same atoms using:
$$L_M^a(t) =\{\phi \in M : C(t) \cap U^c(\phi) \neq \emptyset\}$$

\

In order to use $U^c(\phi)$ instead of $U^c_M(\phi)$ we just need to assume that atoms of different atomized semilattice models that have the same upper constant segments also have the same name. With this choice an atom can be defined by a set $U^c(\phi)$ independetly of other atoms and independetly of the model it belongs to. We can define the atoms by their upper constant segment $U^c(\phi)$ as well as the models by the atoms they have, as follows:

\

\begin{tcolorbox} 
	\textbf{Definiton:} Consider a non-empty subset $U^c(\phi)$ of a set of constants $C$. For any atomized semilattice $M$ over $C$, any set $A$ of atoms that atomizes $M$ and any constant $c$ in $C$, the atom $\phi$ defined by $U^c(\phi)$ is an atom that satisfies $M \Vdash (\phi < c)$ if and only if $\phi \in A$ and $c \in U^c(\phi)$.
\end{tcolorbox}

\

Since atoms are non-empty sets of constants, any atomization of a semilattice model is a hpergraph with the atoms as hyperedges and the constants as vertexes.

\newpage

\noindent
\underline{\textbf{A little on hypergraphs:}}

\

An Hypergraph $H$ is a couple $(V, E)$ where $V =\{v_1, ..., v_n\}$ is a non empty (and generaly finite) set and $E =\{E_1, ..., E_m\}$ is a family of non empty subset of $\mathbb{P}(X)$ (power set of $X$).
\begin{itemize} 
	\item Elements of $V$ are the vertexes of $H$.
	\item Elements of $E$ are the hyperedges of $H$.
\end{itemize}

The incidence matrix is a matrix that fully caracterizes a hypergraph. It is a simply a matrix of size $n \times m$ that verifies $a_{ij} = 1$ if $v_i \in E_j$ and 0 otherwise.

\

\

Axiom AS3 can be rewritten to show the connections between universally defined atoms and models as:

$$(t \leq_M s) \Leftrightarrow \neg \exists \phi : ((\phi \in M) \land (\phi < t) \land (\phi \not < s))$$

where $t$ and $s$ are terms and $t \leq_M s$ means $M \Vdash (t \leq s)$ and the natural homomorphism as been omitted. We add the subscript in $\leq_M$ to highlight the difference with $<$ that compares atoms and terms independetly of the model. 

\

The next theorem allow us to indentify $U^c(\phi)$ with the upper constant segment of $\phi$ in the freest model $F_C(\emptyset)$ as well as $\leq$ with $\leq_{F_C(\emptyset}$.

\

\begin{mybox}{red}{\textbf{Theorem 3}}
	If $A$ is a set of atoms that atomizes $F_C(\emptyset)$ and $\psi$ is any atom, the set $A \cup\{\psi\}$ is also an atomization for $F_C(\emptyset)$.
\end{mybox}

\end{document}
